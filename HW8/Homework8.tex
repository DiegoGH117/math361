\documentclass[12pt]{article}
%%% HEADER: Change at your own risk :) %%%
\usepackage{amsthm,graphicx, dsfont}
\usepackage{epstopdf}
\usepackage{amssymb}
\usepackage{color}
\usepackage{url}
\usepackage{bm}
\usepackage{rotating}
\usepackage{hyperref}
\usepackage{amsmath}

\textwidth = 6.5 in
\textheight = 9 in
\oddsidemargin = 0.0 in
\evensidemargin = 0.0 in
\topmargin = 0.0 in
\headheight = 0.0 in
\headsep = 0.0 in
%%% HEADER (end) %%%%


%%% Start editing from here %%%x
\title{MATH 315, Fall 2020 \\[3pt]
Homework 8
\date{}}
\author{James Nagy}

\begin{document}
\maketitle

\begin{center}
Homework Team \textit{(team number)}: \textit{(team name)} \\
\rule[0pt]{12cm}{1pt}
\end{center}


This assignment is associated with Chapter 5, Differentiation and Integration. It illustrates numerical methods for evaluating the definite integral of a function.
\begin{enumerate}
\item \begin{enumerate}
\item Consider the function $q_2(x)$ which is the polynomial of degree
two that interpolates the function $f(x)$ at the integration interval endpoints
$x = a$, $x=\frac{b-a}{2}$, and $x = b$.  Use the Lagrange form of the quadratic interpolating polynomial to derive a quadrature rule. \textit{(Hint: you will integrate the interpolating polynomial over the interval $[a,b]$.)}\label{rule}
\item What is the name of this quadrature rule? 
\end{enumerate}
\item Evaluate the following integral:
\begin{equation*}
\int^{\pi/2}_0 (8 + 4\cos x) \,dx
\end{equation*}
\begin{enumerate}
\item \label{a} analytically \label{true}
\item using {\tt syms} and {\tt int} in MATLAB)
\item using single application of the trapezoidal rule \label{trap}
\item using single application of the quadrature rule derived in \ref{rule}. \label{rule_app}
\item Use \ref{true} to determine the percent relative error ($\varepsilon_t$) of \ref{trap} and \ref{rule_app}. How can this error be reduced?
\end{enumerate}
\pagebreak
\item 
Write a short MATLAB code that will implement the composite trapezoidal rule using tabular data. Your code should be written as a function m-file, with input data, x and y, and output I, the approximation of the integral. You should make your code robust enough to handle x data that is not equally spaced. The beginning of your code might contain the following:

{\footnotesize
\begin{verbatim}
function I = CompTrap(x,f)
% This function approximates integrates tabular data using the composite
% trapezoidal rule.
% 
%   Input:  x - length n vector of x data
%           y - length n vector of f data
% 
%   Output: I - approximation of the integral
% 
n = length(x);
sum = 0;

for ...

\end{verbatim}
}
\end{enumerate}
\end{document}
