\documentclass[12pt]{article}
%%% HEADER: Change at your own risk :) %%%
\usepackage{amsthm,graphicx, dsfont}
\usepackage{epstopdf}
\usepackage{amssymb}
\usepackage{color}
\usepackage{url}
\usepackage{bm}
\usepackage{rotating}
\usepackage{hyperref}

\textwidth = 6.5 in
\textheight = 9 in
\oddsidemargin = 0.0 in
\evensidemargin = 0.0 in
\topmargin = 0.0 in
\headheight = 0.0 in
\headsep = 0.0 in
%%% HEADER (end) %%%%


%%% Start editing from here %%%x
\title{MATH 315, Fall 2020 \\[3pt]
Homework 3
\date{}}
\author{James Nagy}

\begin{document}
\maketitle

\begin{center}
Homework Team \textit{(team number)}: \textit{(team name)} \\
\rule[0pt]{12cm}{1pt}
\end{center}


This assignment is associated with Chapter 3, Solution of Linear Systems. It illustrates numerical methods for solving linear systems of equations, i.e. $A{x}={b}$, and also some complexities introduced when solving this problem on a computer.  It should also reinforce some of the theory of vector norms.
%
\begin{enumerate}
\item Systems of linear equations have a unique solution when the matrix $A$ is nonsingular, i.e. when $\det(A) \ne 0$. Consider the formula for the determinant of a $2\times 2$ matrix.  
\begin{enumerate}
\item What are two numerical issues introduced in Chapter 2 that may arise when computing this value?
\item Develop two matrices that suffer from these issues.  Assume a floating point system that represents the mantissa with 3 digits.
\end{enumerate}

\item 

\begin{enumerate}
\item Write a MATLAB code that will implement row-oriented forward substitution (see Fig. 3.5).  Your code should be written as a function m-file, with input $A$ (an $n \times n$ matrix) and $b$ (an $n \times 1$ vector), and output the solution $x$ (the $n \times 1$ vector that solves $Ax=b$).

More specifically, the beginning of your code should contain the following:
{\footnotesize
\begin{verbatim}
function x = RowSolve(A,b)
% This function computes the solution of the linear system Ax=b using
% forward substitution, using a row-oriented approach
% 
%   Input:  A - nxn matrix
%           b - nx1 vector
% 
%   Output: x - solution of the linear system
% 

n = length(b);
x = zeros(n,1);

for ...

\end{verbatim}
}

\item Write a MATLAB code that will implement column-oriented forward substitution (see Fig. 3.5).  Your code should be written as a function m-file, with input $A$ (an $n \times n$ matrix) and $b$ (an $n \times 1$ vector), and output the solution $x$ (the $n \times 1$ vector that solves $Ax=b$).
More specifically, the beginning of your code should contain the following:
{\footnotesize
\begin{verbatim}
function x = ColSolve(A,b)
% This function computes the solution of the linear system Ax=b using
% forward substitution, using a column-oriented approach
% 
%   Input:  A - nxn matrix
%           b - nx1 vector
% 
%   Output: x - solution of the linear system
% 

n = length(b);
x = zeros(n,1);

for ...

\end{verbatim}
}

\item The following Matlab statements can be used to explore the computational cost of each function and compare and assess both solutions:
{\footnotesize
\begin{verbatim}
n = 10;
A = tril(rand(n));
b = rand(n,1);

disp('row-oriented:')
tic
xR = RowSolve(A,b);
toc

disp('column-oriented:')
tic
xC = ColSolve(A,b);
toc

disp('infinity norm of difference:')
norm((xC - xR), Inf)

disp('infinity norm of residual:')
norm((b - A*xR), Inf)
\end{verbatim}
}
Create a Matlab script file, called containing the above Matlab statements. Run your script multiple times for various $n$, e.g. $n = 10, 100, 1000, ...$.  Does one approach produce a solution faster than the other?  Does $n$ influence this?  Do the two approaches produce the same results? Is the residual zero? Discuss your findings and rationale for them.  

\item What is a line of code that serves the same function as \texttt{norm((xC - xR), Inf)}?

\end{enumerate}


\item Show that the $\infty$-norm satisfies the properties of a vector norm.



\end{enumerate}
\end{document}
