\documentclass[12pt]{article}
%%% HEADER: Change at your own risk :) %%%
\usepackage{amsthm,graphicx, dsfont}
\usepackage{epstopdf}
\usepackage{amssymb}
\usepackage{color}
\usepackage{url}
\usepackage{bm}
\usepackage{rotating}
\usepackage{hyperref}

\textwidth = 6.5 in
\textheight = 9 in
\oddsidemargin = 0.0 in
\evensidemargin = 0.0 in
\topmargin = 0.0 in
\headheight = 0.0 in
\headsep = 0.0 in
%%% HEADER (end) %%%%


%%% Start editing from here %%%x
\title{MATH 315, Fall 2020 \\[3pt]
Homework 6
\date{}}
\author{NAME: }

\begin{document}
\maketitle

{\bf Important Information:} This will {\bf not} be a team assignment. You must work on this on your own, and turn in your own solutions.  

You should not ask friends or anyone but me for help before Friday's lab. You will have an opportunity to work with a small group in the lab on Friday, and you can then ask that small group questions if you are having trouble. Your lab group can provide help and guidance to each other, but they {\bf should not give} solutions to each other. 

I am happy to provide individual guidance during office hours, so please do not hesitate to ask me for help.

\begin{center}
\rule[0pt]{12cm}{1pt}
\end{center}


This assignment is associated with Chapter 4, Curve Fitting -- we experiment with MATLAB's {\tt polyfit, spline} and {\tt pchip}
functions.
You should write one script m-file to perform all of the experiments and produces all of the plots outlined in the assignment.

\vspace{12pt}

\begin{enumerate}
\item
{\bf Math Problem:}
Suppose $f(x)$ is a function that is $n$ times continuously
    differentiable, and let $p(x)$ be the polynomial of degree $n-1$
    that interpolates $f(x)$ at the points $\{x_i,f(x_i)\}$,
    $i = 1, 2, 3, \cdots, n$.  Then for any $t \in [x_1,x_n]$
    $$
      \left| f(t) - p(t) \right| =
      \left| \frac{f^{(n)}(c)}{n!} \right|
      \left| w(t) \right|,
    $$
    where $c \in [x_1,x_n]$ and 
    $w(t) = (t - x_1) (t - x_2) \cdots (t - x_n)$.

  Show that the maximum error associated with {\bf linear} interpolation
  using equally spaced points is bounded by
  $\displaystyle \frac{M}{8}(x_2-x_1)^2$, where 
  $\displaystyle M = \max_{x_1\leq x \leq x_2} | f''(x)|$. \\[12pt]
  {\footnotesize (Hint: Find the expression for $w(t)$ for linear interpolation, and
  then find the maximum value of $|w(t)|$ on the interval $[x_1, x_2]$.)}
\clearpage
\item
{\bf MATLAB Problem:}
The following is weekly measured laboratory data (the application is not important):

{\footnotesize
\begin{tabular}{|c|c|c|c|c|c|c|c|c|c|c|c|c|c|c|c|c|c|}\hline
Week: & 1 & 2 & 3 & 4 & 5 & 6 & 7 & 8 & 9  & 10 & 11 & 12 & 13 & 14 & 15 & 16 & 17  \\ \hline
Measurement: & 2 & 7 & 5 & 5 & 8 & 9 & 10 & 9 & 15 & 14 & 25 & 45 & 50 & 35 & 48 & 74 & 102 \\ \hline 
\multicolumn{18}{}{c} \\ \cline{1-17}
Week: & 18 & 19 & 20 & 21 & 22 & 23 & 24 & 25 & 26 & 27 & 28 & 29 & 30 & 31 & 32 & 33 \\ \cline{1-17}
Measurement: & 147 & 171 & 199 & 226 & 208 & 200 & 150 & 121 & 54 & 27 & 30 & 12 & 10 & 10 & 4 & 5 \\ \cline{1-17}
\end{tabular}
}

\begin{enumerate}
\item
Use {\tt polyfit} and {\tt polyval} to construct an interpolating polynomial
for {\em all} of the given data.  Create a plot that contains the data points
as red circles and the interpolating polynomial as a blue curve.   
Is the
interpolating polynomial a good representation of the data?  Explain.
\item
You get a warning about that suggests trying ``centering and scaling". This only requires a small 
modification of calls to {\tt polyfit} and {\tt polyval}. Try, and comment about whether this 
improves your result.
\item
Use {\tt polyfit} and {\tt polyval} to construct an interpolating polynomial
using data for weeks $1, 3, 5, \ldots, 33$.  Create a plot that contains 
{\em all} data points in the above table as red circles, and the interpolating polynomial 
through data for weeks $1, 3, 5, \ldots, 33$.  Is the interpolating polynomial
a good representation of the data?  Explain.
\item
Use {\tt polyfit} and {\tt polyval} to construct an interpolating polynomial
using data for weeks
$$
1, 2, 3, 4, 6, 9, 11, 14, 17, 20, 23, 25, 28, 30, 31, 32, 33
$$
Create a plot that contains 
{\em all} data points in the above table as red circles, and the interpolating polynomial 
through data for the weeks specified in this problem.  Is the interpolating polynomial
a good representation of the data?  Why do we get a better interpolating polynomial
when using the data for this problem than when using the data in the previous
problem? 
\item
Repeat the above experiments, but this time use MATLAB's built-in {\tt spline} function. (You will not need centering and scaling for this).
Does the use of {\tt spline} change the results?
\item
Repeat the above experiments, but this time use MATLAB's built-in {\tt pchip} function.   (You will not need centering and scaling for this).

Does the use of {\tt pchip} change the results? 
\end{enumerate}
\end{enumerate}

Turn in all codes, as well as any plots and tables
that you need to clearly explain your results.  Your work should
be neatly organized.
\end{document}
